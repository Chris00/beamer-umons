\documentclass[compress]{beamer}

\usepackage[utf8]{inputenc}
\usepackage{amssymb}

\newcommand{\IR}{\mathbb{R}}


\usetheme[navigation]{UMONS}
%\usetheme[navigation,totalframenumber=false]{UMONS}

\title{Exemple de transparents UMONS}
\subtitle{(pour beamer)}
\author{C. Troestler}
\institute[(Math)]{%
  Institut de Mathématique\\
  Université de Mons
  \\[2ex]
  \includegraphics[height=4ex]{UMONS}\hspace{2em}%
  \raisebox{-1ex}{\includegraphics[height=6ex]{UMONS_FS}}
}

\begin{document}

\begin{frame}[plain]
  \titlepage
\end{frame}

\begin{frame}
  \tableofcontents
\end{frame}

\section{Première section}
\subsection{Première sous-section}
\begin{frame}
  \frametitle{Premier transparent}

  \begin{itemize}
  \item Niveau 1 d'\texttt{itemize}
    \begin{itemize}
    \item Niveau 2 d'\texttt{itemize}
    \item Niveau 2 d'\texttt{itemize}
    \end{itemize}
  \item Niveau 1 d'\texttt{itemize}
  \end{itemize}
\end{frame}

\subsection{Deuxième sous-section}
\begin{frame}
  \frametitle{Second transparent}

  \begin{theorem}
    Soit $f : \IR \to \IR$ une fonction continue et coercive.  Alors
    $f$ possède un minimum.
  \end{theorem}
\end{frame}


\section{Deuxième section}
\begin{frame}
  \frametitle{Second transparent}

  \begin{enumerate}
  \item Niveau 1 d'\texttt{enumerate}
    \begin{enumerate}
    \item Niveau 2 d'\texttt{enumerate}
    \item Niveau 2 d'\texttt{enumerate}
    \end{enumerate}
  \item Niveau 1 d'\texttt{enumerate}
  \end{enumerate}
\end{frame}


\end{document}
%%% Local Variables: 
%%% mode: latex
%%% TeX-master: t
%%% End: 
